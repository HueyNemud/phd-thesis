\documentclass[a4paper,10pt]{article}
\usepackage[utf8]{inputenc}
\usepackage[frenchb]{babel}

%opening
\title{}
\author{}
    \setcounter{secnumdepth}{5}
    \setcounter{tocdepth}{5}
\begin{document}

\maketitle

\begin{abstract}
\end{abstract}

\tableofcontents

\subsection{Introduction}
\subsubsection{Le 'spatial turn' en Histoire}
Ici on pose le contexte de l'utilisation de l'espace en Histoire via le 'spatial turn' (Lefebvre, Cosgrove ,Desmarais, etc).
C'est aussi l'occasion d'introduire quelques travaux fondamentaux sur l'étude de l'espace, comme le travail sur les Halles.
\subsubsection{les besoins de représentation et de modélisation pour les études complexes}
Etat des lieux et limites des usages actuels par l'historien ainsi que les besoins énoncés (Grégory, Rumsey, Noizet,etc.).
\subsubsection{Problématique de la thèse et objectifs}
\subsubsection{Plan du mémoire}


\section{Partie 1 : L'espace géographique, un outil pour l'analyse historique}
\subsection{L'information géographique en Histoire}
Classification des travaux en en Histoire traitant de (analyse morphologique) ou s'appuyant sur (analyse spatiale) l'espace, 
comme décrit par des auteurs tels Arnaud.
\subsubsection{Usages de l'information géographique en Histoire}
\paragraph{L'espace comme support}
Ici on parlera des études historiques utilisant l'espace pour étudier des pratiques sociales, des échanges (Arnaud).
\paragraph{L'espace comme objet d'étude}
Présenter les deux aspects de cet usage :
\begin{itemize}
 \item études morphologiques (historiographie descriptive).
 \item dynamiques de l'espace
\end{itemize}
Sur les dynamiques, introduire les concepts de saillances, de transformation (Lévy, Cheylan, Rodier/Lefebvre)

\subsubsection{Les sources d'informations spatiales}
\paragraph{Les relevés archéologiques / le terrain}
\paragraph{Les textes}
\paragraph{Les gravures et iconographies}
\paragraph{Les sources cartographiques}
C'est l'occasion ici de définir correctement les différentes sources cartographique (plans, cartes, cadastres vs parcellaire, etc...),
ainsi que l'évolution de leur forme et de leur contenu (par ex, d'après Pinon).


\subsubsection{Les particularités des sources cartographique historiques}
\paragraph{Hétérogénéité des sources}
\subparagraph{La carte n'est pas une photographie de la réalité mais répond à une vision et à des objectifs.}
\subparagraph{Hétérogénéité due aux processus de cartographie (construction du plan, recopie).}

\paragraph{La source cartographique est sources d'incertitudes}
Définir les différentes classes de l'incertitude (Le Ber, De Runz, etc.)
\subparagraph{Spatiales : lacunes, imprécision.}
\subparagraph{Temporelles : incertitudes, vague}
\subparagraph{Sémantique : conflit, incertitude, lacunes}

Enfin, définir la carte non comme une 'coupe' dans le temps mais comme un objet dynamique à part entière: une carte à 
une profondeur temporelle (l'exemple des minutes du plan de Verniquet, des récupérations du plan par d'autres plus tardifs.)

\subsection{Les outils SIG pour l'étude historique}

\subsubsection{Les points d'interêt des SIG pour l'Hisoire}
\paragraph{Visualisation}
\paragraph{Modélisation}
\paragraph{Analyse}

\subsubsection{Etat de l'art des SIG historique}
\paragraph{Les applications existantes}
\paragraph{Analyse comparative des modèles selon les critères: de visu cartographique, de généricité, prise en compte des dynamiques (spatio-temporel),
incertitudes.}


\subsubsection{Les SIG et la représentation de phénomènes dynamiques}
Présenter l'inadéquation des SIG pour la représentation de phénomènes dynamiques et les difficultés
rencontrées par ceux qui essayent de les utiliser dans ce but (Rumsey, Peuquet, etc.).\\
Présenter également le problème de la temporalité des sources historiques, que l'on retrouve aussi
en archéologie mais qui est absent de la plupart des propositions (Peuquet)

\subsection{Présentation de la zone d'étude}
\subsubsection{Les espaces considérés}
Paris, le quartier Saint-Martin : choix et spécificités
\subsubsection{Temporalités}
Quelles sont les temporalités choisies, quelles sont leurs particularités et quelles sont les conséquences d'un
tel choix sur notre travail.
\subsubsection{Sources historiques choisies}
Quelles sont les sources choisies, quelles sont leurs spécificités.

\subsection{Conclusion de partie : la nécessité d'un processus adapté à l'utilisation de données historiques au sein 
de SIG temporels}

\section{Une méthode de construction d'objets géo-historiques à partir de sources cartographiques anciennes}
\subsection{Introduction : Présentation globale de la méthode}

\subsection{Géoréférencer les cartes anciennes}
\subsubsection{Introduction}
Préciser l'importance d'un géoréférencement approprié aux cartes anciennes,
 voir à chaque carte en particulier (Baletti, Bitelli, Grosso, etc.)
\subsubsection{Etat de l'art : méthodes et applications}

\subsubsection{Application de deux méthodes aux plans de Paris}
\paragraph{Présentation des méthodes choisies}
\paragraph{Application aux plans}

\subsubsection{Mesures de qualité du géoréférencement}
\paragraph{Pertinence des mesures (déformations, erreurs,etc.)}
\paragraph{Mesures et cartographiques des déformations des plans de Paris}

\subsubsection{Conclusion sur les méthodes appropriées}
Distinction selon l'objectif visé la méthode approprié sur Paris.
Tracer les limites de ces méthodes sur des temps plus anciens (<=17e) où le souci de précision géométrique (ou des modes
de représentation) ne permet pas de géoréférencement correct : jusqu'où cela vaut-il le coup de 'forcer' le géoréférencement?

\subsection{Une base de données historique spatiale et temporelle}
\subsubsection{Introduction}
Expliciter ici l'importance du temps pour analyser l'information spatiale historique, et 
le besoin de conserver la spécificité des sources historiques.

\subsubsection{Etat de l'art : bases de données temporelles et spatio-temporelles}

\subsubsection{Proposition : base de données historique}
\paragraph{Structuration des sources historiques}
La source historique est elle-même un objet historique : temporalisation de la source
\paragraph{Des sources historiques aux objets historiques}
Utilité des données vecteur, temporalisation des données (probabilité d'existence et possibilité d'existence).\\
\paragraph{Instanciation de la base de données}

\subsubsection{Proposition : un graphe pour représenter les dynamiques d'évolution}
\paragraph{Lier les objets historiques entre eux}
\subparagraph{Objectifs et particularités (évolution discrète, l'hétérogénéité des représentations, des granularités et des échelles) }
\subparagraph{Contraintes sur les liens : cohérence des sources, cohérence temporelle. }
\paragraph{Un graphe de filiation}
\subparagraph{Formalisation du graphe}
\subparagraph{Instanciation du graphe sur un exemple.}
\paragraph{Conclusion : apports et limites}
C'est le moment de pointer le problème de la construction du graphe, mais aussi des avantages apportés par
la structuration simple qui veut que l'on 'colle aux sources'.
\subsection{Une méthode de construction automatique de graphes de filiations}
\subsubsection{Introduction}
Justifier la construction automatique et la difficulté, présenter le parallèle avec
les méthode d'appariement géographiques et le choix de se diriger vers une métaheuristique.

\subsubsection{Etat de l'art : méthode d'appariement et métaheuristiques}

\subsubsection{Proposition}
\paragraph{Présentation de l'approche multicritères}
\subparagraph{Introduction de la notion de vraisemblance d'un graphe de filiation}
\subparagraph{Mesurer la vraisemblance : Aggrégation de critères vs multicritères}
\subparagraph{Distances et similarités : sur les 3 axes, présenter les mesures possibles et choisies ou proposées}
\subparagraph{Différencier filiation et équivalence}

\paragraph{Un recuit simulé pour la construction du graphe de filiations}
\subparagraph{Justification du choix du recuit simulé}
\subparagraph{Formalisation et expression énergétique.}
\subparagraph{Réduction de l'espace de recherche}

\paragraph{Application}
\subparagraph{Points : critère et résultats (adresses)}

\subparagraph{lignes : critère et résultats (réseaux)}

\subparagraph{surfaces : critère et résultats (parcelles)}


\paragraph{Estimer la qualité de la construction}
\subparagraph{ Qualité interne : qualité de la méta-heuristique (convergence, sensibilité,etc.)}

\subparagraph{Qualité externe : qualité de l'appariement réalisé (précision/rappel, expert, etc.)}

\subparagraph{Confrontation avec d'autres méthodes }



\subsubsection{Conclusion}


\subsection{Premières exploitations du des graphes construits}
\subsubsection{Construire un référentiel historique}
Comment le graphe peut servir à faire de la conflation de données.

\subsubsection{Cartographies : transformations, rythmes}
\subsubsection{Géocodage}
\paragraph{Méthode ad-hoc}
\paragraph{Par appariement à partir du graphe.}

\section{Conclusion}
\subsubsection{Synthèse}
\subsubsection{Perspectives}

\end{document}
